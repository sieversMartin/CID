% !TeX encoding = UTF-8
% !TeX program = pdflatex
% !BIB program = biber
% !TeX spellcheck = de_DE

%%% Um einen Artikel auf deutsch zu schreiben, genügt es die Klasse ohne
%%% Parameter zu laden.
\documentclass[]{cidarticle}
%%% To write an article in English, please use the option ``english'' in order
%%% to get the correct hyphenation patterns and terms.
%%% \documentclass[english]{cidarticle}
%%%
%%% Die CID-Klasse nutzt biblatex, so dass man die .bib-Datei angeben muss:
\addbibresource{cidarticle-example.bib} %Beispiel-Bibliography

\usepackage{blindtext} %nur zu Testzwecken
%%
\begin{document}
%%% Mehrere Autoren werden durch \and voneinander getrennt.
%%% Die Fußnote enthält die Adresse sowie eine E-Mail-Adresse.
%%% Das optionale Argument (sofern angegeben) wird jeweils für das
%%% Inhaltsverzeichnis *und* den Kolumnentitel im Kopf der Seite verwendet.
%%% Die Angaben für den Kolumnentitel können mittels optionalen Arguments
%%% für \title bzw. \author gesetzt werden.
\title[Ein Kurztitel wenn der Titel z.B. Fußnoten enthält]{Ein sehr langer Titel über mehrere Zeilen mit sehr vielen
Worten und noch mehr Buchstaben}
%%%\subtitle{Untertitel / Subtitle} % if needed
\author[1]{Vorname1 Nachname1}{vorname.name@affiliation1.de}{0000-0000-0000-0000}
\author[2]{Firstname2 Lastname2}{vorname.name@affiliation2.de}{0000-0000-0000-0000}
\author[3]{Firstname3 Lastname 3}{vorname.name@affiliation1.de}{0000-0000-0000-0000}
\author[1]{Firstname4 Lastname 4}{vorname.name@affiliation1.de}{0000-0000-0000-0000}
\affil[1]{Universität\\Abteilung\\Straße\\Postleitzahl Ort\\Land}
\affil[2]{University\\Department\\Address\\Country}
\affil[3]{University\\Department\\Address\\Country}

\maketitle

\begin{abstract}
Die Zusammenfassung sollte etwa 70 bis 150 Worte umfassen und besteht in der Regel aus einem Absatz.
\end{abstract}
\begin{keywords}
Schlagwort1 \and Schlagwort2 %Keyword1 \and Keyword2
\end{keywords}
%%% Beginn des Artikeltexts
\section{Überschrift der Ebene 1}
\blindtext[2]

\subsection{Referenzen auf Einträge in der Bibliografie}
Jetzt ein paar Referenzen auf \cite{Bernhard2017} oder auch \cite{ANKOM2014} sowie auf eine Liste \cites[17]{Anderson2012}{Berges2013}{Anderson2001}.

\subsection{Grafiken}
Hier kommt eine Grafik:
\begin{figure}
   \includegraphics[width=\textwidth]{example-image-a}
   \caption{Eine Beschriftung}
   \label{fig:Beispielgrafik}
\end{figure}

Auf die Grafik kann man dann verweisen: \cref{fig:Beispielgrafik}.

\subsection{Mehr Text}
\blindtext[18]

\subsection{Überschrift der Ebene 2}
\blindtext[5]

%%% Ausgabe der Bibliographie (biber vorher aufrufen)
\printbibliography
\end{document}
